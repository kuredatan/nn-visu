\documentclass[10pt,frenchb]{beamer}

\usepackage[utf8]{inputenc}
\usetheme[progressbar=frametitle]{metropolis} 
\usepackage{appendixnumberbeamer}

\usepackage{graphicx}
\usepackage{wrapfig}
\usepackage{xcolor}            
\usepackage{booktabs}
\usepackage[scale=2]{ccicons}

\usepackage{pgfplots}
\usepgfplotslibrary{dateplot}


\usefonttheme{professionalfonts}
\usepackage{times}
\usepackage{tikz}
\usepackage{amsmath,amssymb}
\usepackage{verbatim}
\usetikzlibrary{arrows,shapes}	
\pgfplotsset{compat=1.4}

\author{Cl\'{e}mence R\'{e}da \& XiaoQi Xu}
\title[Visualizing and Interpreting]{Visualizing and Interpreting Neural Networks}
\date{January $14^{\text{th}}$, 2019}
 
\begin{document}
\maketitle

%15'

\section{Introduction}

%Reusing material / figures / slides from other people. You can take figures from papers or other people’s slides to illustrate an algorithm or explain a method. However, always properly acknowledge the source if you do so.

\subsection{Topic}

\begin{frame}
\frametitle{Introduction}

%TODO Understand Neural Networks + using Deconvnets => Research paper
%(i) read and understand a research paper,

\end{frame}

\begin{frame}{Goal}

%TODO interpreting neural networks

\end{frame}

\begin{frame}
  \tableofcontents
\end{frame} 

\section{Our Work}

\subsection{Experiments}

\begin{frame}

%TODO protocole
%(ii) implement (a part of) the paper, and 

\end{frame}

\begin{frame}

%TODO données, architecture, etc.

\end{frame}

\section{Results}

\subsection{Qualitative Results}

% TODO When describing results, please show both qualitative and quantitative results you have obtained and any interesting observations / findings you have made.

\begin{frame}

%TODO

\end{frame}

\subsection{Quantitative Results}

\begin{frame}

%8 minutes talk + 2 minutes questions
%To minimize the time for switching between presentations, we ask you to submit a link to your presentation. We will show the slides for all projects from our computer. This link should therefore be publicly accessible. We suggest that you prepare your presentations with Google Slides, or upload a PDF file on any online platform.

%46	16/01/2019	2	14:10	14:20	Clémence Réda, Xiaoqi Xu	1	J	Mihai	Josef

%TODO perform qualitative/quantitative experimental evaluation.

\end{frame}

\section{Conclusion}

 \begin{frame}

%TODO

\end{frame}

\begin{frame}[allowframebreaks]{References}
\bibliography{./latex/egbib} 
\bibliographystyle{apalike}
\end{frame}
\end{document}

%- a CNN with relatively low number of parameters (20k) should easily surpass 70% acc.
%- deeper CNNs (100k-1M parameters) should have ~80% acc. without data augmentation and 85+% acc. with

%2. The quantitative evaluation (on the associated test sets) of your architectures should also play an important role in the project! Look into each class' statistics / the confusion matrix / etc and comment them. Compare your results with the state of the art in the case of CIFAR 10/100 - http://rodrigob.github.io/are_we_there_yet/build/classification_datasets_results.html.



